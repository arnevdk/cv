\documentclass{resume}


\begin{document}

{\Large dr. ir. Arne Van Den Kerchove}
\smallskip

\texttt{arne@vandenkerchove.com}\sep
\texttt{+32 473 32 78 71}\sep
Leuven, Belgium\\
\texttt{orcid.org/0000-0002-9367-2986}\sep
\texttt{linkedin.com/in/arnevdk}\sep
\texttt{github.com/arnevdk}

\section{About me}

\textbf{Leveraging AI to enhance patient welfare.}
\smallskip

Biomedical engineer and postdoctoral researcher specializing in machine learning
for neurotechnology.
Experienced in brain-computer interfacing (BCI) methods and applications, EEG data classification, and real-time biosignal processing.
Strong interdisciplinary background in computational neuroscience, clinical
research, and software development.


\section{Education}
\cvitem{2024}{Ph.D. in Biomedical Sciences}{KU Leuven}
\smallskip

\cvitem{2024}{Ph.D. in Control Science and Signal Processing}{University of Lille}
\smallskip

\cvitem{2020}{M.Sc. in Engineering Sciences: Computer Science (spec. AI)}{KU Leuven}
\smallskip

\cvitem{2017}{B.Sc. in Informatics}{KU Leuven}

\section{Experience}
\cvitem{2025}{Project lead}{MindSpeaker BCI}
\begin{itemize}
	%\item Leading the development of an EEG-BCI assistive and augmentative
	%      communication platform for patients with eye motor impairment.
	%\item Overseeing clinical and scientific validation of assistive and
	%      augmentative BCI communication devices and protocols.
	\item Leading the development and validation of an EEG-based dementia screening product.
	\item Overseeing clinical and scientific validation of assistive and augmentative BCI communication devices and protocols.
\end{itemize}
\bigskip

\cvitem{2020 - ongoing}{Researcher computational neuro\-science}{University of Lille, KU Leuven}
\begin{itemize}
	\item Developed state-of-the-art classification algorithms for event-related
	      EEG data. Enhanced machine learning methods for various BCI
	      paradigms, automated EEG-based dementia screening, and fMRI phenotype prediction.
	\item Designed software for real-time bio\-signal processing and
	      brain-computer interfacing.
	\item Performed 3 multi-center clinical studies in patient and control
	      groups to validate BCI efficacy and user experience.
	\item Supervised 6 master's students working on various aspects of BCI development
	      and automated dementia screening, and served for 4 years as teaching assistant
	      for BCI design and signal processing classes.
	\item Conducted interdisciplinary research in an international context with
	      multiple academic, clinical and end-user stakeholders and industry partners.
	\item Published research findings in peer-reviewed scientific journals and presented at
	      4 international conferences.

\end{itemize}
\bigskip

\cvitem{2022 - ongoing}{BCI project supervisor}{NeuroTech Leuven}
\begin{itemize}
	\item Guided 6 student teams participating in the annual Neuro\-TechX BCI
	      competition and various other neurotechnology research projects
	      on scientific, technical, and project management topics.
	\item Helped kick\-start and grow a vibrant community of students, researchers,
	      and industry partners passionate about neuro\-technology.
\end{itemize}
\bigskip

\cvitem{2020}{Teaching assistant \emph{Fundamentals of Computer Science}}{KU Leuven}
\begin{itemize}
	\item Served as teaching assistant for bachelor-level classes in algorithmic reasoning and python programming.
\end{itemize}
\bigskip

\cvitem{2014 - 2020}{Freelance full-stack web developer}{self employed}

\section{Skills}
\paragraph{Technical skills:}
neural and biosignal analysis,
real-time signal processing,
machine learning and deep learning,
data science \& statistics,
software and full-stack development,
accelerated \& scientific computing, electrophysiology recording.

\paragraph{Research \& development skills:} neurophysiology, clinical research, interdisciplinary
collaboration, experimental study design, scientific writing, UX/UI design and evaluation.

\paragraph{Soft skills:} team player, analytical, creative, problem-solving,
result-oriented.

\paragraph{Languages:} Dutch (\emph{native}), English (\emph{fully proficient}), French
(\emph{advanced}), German (\emph{intermediate}).

\pagebreak

%\section{Selected publications and projects}
%\subsubsection*{Peer-reviewed journal articles}
%\begin{description}
%	\item  Van Den Kerchove \emph{et al.}, (2024) \\
%	      \emph{Correcting for ERP latency
%		      jitter improves gaze-independent BCI decoding}, \\
%	      Journal of Neural Engineering.
%	\item Van Den Kerchove \emph{et al.}, (2022) \\
%	      \emph{Classification of event-related potentials with regularized spatiotemporal LCMV beamforming}, \\
%	      Applied Sciences.
%	\item Libert \emph{et al.}, (2022) \\
%	      \emph{Analytic beamformer transformation for transfer learning in motion-onset visual evoked potential decoding}, \\
%	      Journal of Neural Engineering.
%\end{description}
%
%\subsubsection*{Ongoing and other projects}
%
%\begin{description}
%	\item Verovnik \emph{et al.}, (2025)\\
%	      \emph{Phenotype prediction from resting state fMRI connectivity}.
%	\item Van Dyck, Van Den Kerchove \& Van Hulle, (2025)\\
%	      \emph{An open-source implementation of a closed-loop electrocorticographic
%		      brain-computer interface with Micromed, Fieldtrip, and Psychopy}, \\
%	      Biomedical Signal Processing and Control (submitted).
%	\item Van Den Kerchove \emph{et al.}, (2025) \\
%	      \emph{Block-Term Tensor Discriminant
%		      Analysis for brain-computer interface decoding}.
%	\item Van Den Kerchove \emph{et al.}, (2025) \\
%	      \emph{Diagnosing Alzheimer's Disease and Frontotemporal Dementia from
%		      resting state EEG using Riemannian Geometry}.
%	\item Van Den Kerchove \emph{et al.}, (2025) \\
%	      \emph{Case studies on visual ERP-based brain-computer interface use by
%		      individuals with severe physical, speech and eye movement impairment}, \\
%	      Journal of NeuroEngineering and Rehabilitation (submitted).
%	\item Van Den Kerchove \emph{et al.}, (2025) \\
%	      \emph{ An event-related potential BCI speller using a wearable, single-channel
%		      EEG headset with electrodes on the forehead}, \\
%	      Proceedings of the International Work-Conference on Artificial Neural Networks.
%	\item Van Den Kerchove (2024) \\
%	      \emph{A visual brain-computer interface for gaze-free communication}, \\
%	      doctoral thesis, KU Leuven.
%\end{description}
\end{document}

%    I am able to analyse, organise and present large volumes of information in a clear manner;
%    I am able to collect and analyse complex data sets and to present results concisely
%    I am a problem-solver, innovative, creative
%    I am flexible and can work to deadlines
%    I can work autonomously but also in a team
