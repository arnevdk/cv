\documentclass{resume}


\begin{document}

{\Large dr. ir. \textsc{Arne Van Den Kerchove}}
\smallskip

Leveraging AI and neurotechnology to improve patient welfare.
\smallskip

\texttt{arne@vandenkerchove.com} \\
\texttt{+32 473 32 78 71} \\
\texttt{\url{https://orcid.org/0000-0002-9367-2986}} \\
\texttt{\url{https://linkedin.com/in/arnevdk}}

\section{Education}
\cvitem{2024}{Ph.D. in Biomedical Sciences}{KU Leuven}
\cvitem{2024}{Ph.D. in Control Science and Signal Processing}{University of Lille}
\cvitem{2020}{M.Sc. in Engineering Science: Computer Science (spec. AI)}{KU Leuven}
\cvitem{2017}{B.Sc. in Informatics}{KU Leuven}

\section{Experience}
\cvitem{2025 - ongoing}{Project lead}{MindSpeaker BCI}
\begin{itemize}
	\item Developing an EEG-BCI assistive and augmentative
	communication platform for patients with eye motor impairment.
	\item Clinical and scientific validation of assistive and
	augmentative BCI communication devices and protocols.
\end{itemize}
\cvitem{2020 - ongoing}{Researcher Computational Neuroscience}{KU Leuven, University of Lille}
\begin{itemize}
 \item Developed state-of-the-art classification algorithms for event-related
 EEG data.
 \item Developed and enhanced machine learning methods for various BCI
 paradigms and automated EEG-based dementia screening.
 \item Developed software for real-time biosignal processing and
 brain-computer interfacing.
 \item Performed multi-center clinical studies in patients and control
 populations to validate BCI efficacy and user experience.
 \item Supervised master students working on various aspects of BCI development
 and automated dementia screening; teaching assistant for brain-computer
 interfacing classes.
 \item Performed interdisciplinary research in an international context with
 multiple academic, clinical and end-user stakeholders.
 \item Presented at international conferences and published in peer-reviewed
   journals.
\end{itemize}
\cvitem{2022 - ongoing}{BCI project supervisor}{NeuroTech Leuven}
\begin{itemize}
	\item Project and technical advisor for extracurricular neurotechnology
	student teams competing in the annual NeuroTechX brain-computer
	interfacing competition.
\end{itemize}
\cvitem{2020}{Teaching assistant Fundamentals of Computer Science}{KU Leuven}
\smallskip

\cvitem{2014 - 2020}{Freelance full-stack web developer}{self employed}

\section{Selected publications}
\begin{description}
  \item Van Den Kerchove \emph{et al.}, (2025). \textbf{Block-Term Tensor Discriminant
  Analysis for brain-computer interface decoding}, \emph{in preparation}.
  \item Van Den Kerchove \emph{et al.}, (2025). \textbf{Diagnosing Alzheimer's Disease
  and Frontotemporal Dementia from resting state EEG using Riemannian
  Geometry}, \emph{in preparation}.
  \item Van Den Kerchove (2024). \textbf{A visual brain-computer interface for gaze-free
  communication}, \emph{doctoral thesis}.
  \item Van Den Kerchove \emph{et al.}, (2024). \textbf{Correcting for ERP latency
  jitter improves gaze-independent BCI decoding}, \emph{Journal of Neural
  Engineering}.
  \item Van Den Kerchove \emph{et al.}, (2022). \textbf{Classification of event-related
  potenials with regularized spatiotemporal LCMV beamforming}, \emph{Applied Sciences}.
\end{description}

\section{Skills}
\paragraph{Key competences:} Electroencephalography, neural and biosignal processing,
machine learning and data science,
software and full-stack development,
technical and scientific writing,
interdisciplinary experimental research.
\paragraph{About me:} analytical, solution oriented,
\paragraph{Languages:} Dutch (native), English (fully proficient), French
(advanced), German (intermediary)
\end{document}
