\documentclass{resume}


\begin{document}

{\Large dr. ir. \textsc{Arne Van Den Kerchove}}
\smallskip

\texttt{arne@vandenkerchove.com}\sep
\texttt{+32 473 32 78 71}\sep
Leuven, Belgium\\
\texttt{orcid.org/0000-0002-9367-2986}}\sep
\texttt{linkedin.com/in/arnevdk}}\sep
\texttt{arne.vandenkerchove.com}}

\section{About me}

\textbf{Leveraging AI to enhance patient welfare.}
\smallskip

Biomedical engineer and researcher specializing in machine learning and neurotechnoloy.
Experienced in developing brain-computer interface (BCI) applications, machine
learning for EEG data classification, and real-time biosignal processing.
Strong interdisciplinary background in computational neuroscience, clinical
research, and software development.


\section{Education}
\cvitem{2024}{Ph.D. in Biomedical Sciences}{KU Leuven}
\smallskip

\cvitem{2024}{Ph.D. in Control Science and Signal Processing}{University of Lille}
\smallskip

\cvitem{2020}{M.Sc. in Engineering Science: Computer Science (spec. AI)}{KU Leuven}
\smallskip

\cvitem{2017}{B.Sc. in Informatics}{KU Leuven}

\section{Experience}
\cvitem{2025 - ongoing}{Project lead}{MindSpeaker BCI}
\begin{itemize}
	\item Leading the development of an EEG-BCI assistive and augmentative
	communication platform for patients with eye motor impairment.
	\item Overseeing clinical and scientific validation of assistive and
	augmentative BCI communication devices and protocols.
\end{itemize}
\bigskip

\cvitem{2020 - ongoing}{Researcher Computational Neuroscience}{KU Leuven, University of Lille}
\begin{itemize}
 \item Developed state-of-the-art classification algorithms for event-related
 EEG data.
 \item Enhanced machine learning methods for various BCI
 paradigms and automated EEG-based dementia screening.
 \item Developed software for real-time biosignal processing and
 brain-computer interfacing.
 \item Performed multi-center clinical studies in patients and control
 populations to validate BCI efficacy and user experience.
 \item Supervised 6 master's students working on various aspects of BCI development
 and automated dementia screening and served as teaching assistant for BCI classes.
 \item Conducted interdisciplinary research in an international context with
 multiple academic, clinical and end-user stakeholders.
 \item Published research findings in peer-reviewed journals and presented at international conferences.

\end{itemize}
\bigskip

\cvitem{2022 - ongoing}{BCI project supervisor}{NeuroTech Leuven}
\begin{itemize}
  \item Provided technical and project guidance to student teams participating in the NeuroTechX BCI competition.
\end{itemize}
\bigskip

\cvitem{2020}{Teaching assistant Fundamentals of Computer Science}{KU Leuven}
\begin{itemize}
  \item Served as teaching assistant for classes in algorithmic reasoning and
    python programming concepts.
\end{itemize}
\bigskip

\cvitem{2014 - 2020}{Freelance full-stack web developer}{self employed}

%\section{Selected publications}
%\begin{description}
%  \item Van Den Kerchove \emph{et al.}, (2025). \textbf{Block-Term Tensor Discriminant
%  Analysis for brain-computer interface decoding}, \emph{in preparation}.
%  \item Van Den Kerchove \emph{et al.}, (2025). \textbf{Diagnosing Alzheimer's Disease
%  and Frontotemporal Dementia from resting state EEG using Riemannian
%  Geometry}, \emph{in preparation}.
%  \item Van Den Kerchove (2024). \textbf{A visual brain-computer interface for gaze-free
%  communication}, \emph{doctoral thesis}.
%  \item Van Den Kerchove \emph{et al.}, (2024). \textbf{Correcting for ERP latency
%  jitter improves gaze-independent BCI decoding}, \emph{Journal of Neural
%  Engineering}.
%  \item Van Den Kerchove \emph{et al.}, (2022). \textbf{Classification of event-related
%  potenials with regularized spatiotemporal LCMV beamforming}, \emph{Applied Sciences}.
%\end{description}

\section{Skills}
\paragraph{Technical expertise:} EEG recording \& processing, biosignal
analysis, machine learning, data science, real-time signal processing, software
and full-stack development, scientific computing,

\paragraph{Reasearch \& development:} clinical research, interdisciplinary
collaboration, experimental study design, scientific computing.

\paragraph{Soft skills:} analytical, innovative, problem-solving,
result-oriented.

\paragraph{Languages:} Dutch (native), English (fully proficient), French
(advanced), German (intermediary)

\end{document}



%    I am able to analyse, organise and present large volumes of information in a clear manner;
%    I am able to collect and analyse complex data sets and to present results concisely
%    I am a problem-solver, innovative, creative
%    I am flexible and can work to deadlines
%    I can work autonomously but also in a team
