\documentclass{resume}
\usepackage{doi}
\usepackage[style=apa, sorting=ynt,doi=false]{biblatex}
%\usepackage{enumitem}
\usepackage{makecell}
\usepackage{tabularx}
%\usepackage{titlesec}
%\usepackage{fontspec}

%\usepackage{academicons}
%\usepackage{hanging}
%\usepackage{fontawesome}
%\usepackage{amsmath}


% Structure

\newcommand{\cventry}[4]{
	\begin{tabularx}{\linewidth}{@{}p{.9in} X@{}}
		#1 & \makecell[{{X}}t]{\textbf{#2}, #3 \\#4}
	\end{tabularx}\smallskip\smallskip
}


% Layout

%\geometry{left=1in,right=1in,top=1in,bottom=1in}
\pagenumbering{gobble}

%Preamble

\title{\textsc{Curriculum Vit\ae}}
\date{}
\addbibresource{publications.bib}

\begin{document}
\maketitle

{\Large\textbf{Arne Van Den Kerchove}, Ph.D.}\\
Postdoctoral researcher computational neuroscience \\
KU Leuven, Laboratory for Neuro- and Psychophysiology
\bigskip

\texttt{arne.vandenkerchove@kuleuven.be} \\
\texttt{arne@vandenkerchove.com} \\
\texttt{+32 473 32 78 71} \\
\texttt{\url{orcid.org/0000-0002-9367-2986}} \\
\texttt{\url{linkedin.com/in/arnevdk}}\\
\texttt{\url{github.com/arnevdk}}


\section*{About me}

Biomedical engineer and postdoctoral researcher specializing in machine learning for neurotechnology. Experienced in
brain-computer interfacing (BCI) methods and applications, EEG data classification, and real-time biosignal processing.
Strong interdisciplinary background in computational neuroscience, clinical research, and software development.

\section*{Education}

\cventry{2024}{Ph.D. in Biomedical Sciences}{KU Leuven}{Supervisor: prof.
	Marc Van Hulle}
\cventry{2024}{Ph.D. in Control Science and Signal Processing}{University of
	Lille}{Supervisors: prof. François Cabestaing, prof. Hakim Si-Mohammed}
\cventry{2020}{M.Sc. in Engineering Science: Computer Science}{KU Leuven}{
	%Thesis: \textit{Linguistic transcription of EEG responses to sequences of visual stimuli}                \\
	Supervisor: prof. Marc Van Hulle                                                                         \\
	Major: Artificial Intelligence                                                                           \\
	Degree: \textit{cum laude}                                                                               \\
}
\cventry{2017}{B.Sc. in Informatics}{KU Leuven}{
	% Thesis: \textit{Development of a quadcopter simulator and autopilot with stereoscopic object detection} \\
	Minor: Natural Sciences
}

\section*{Publications \& projects}

%\nocite{*}
%\printbibliography[heading=none]

\subsubsection*{Peer-reviewed journal articles}
\begin{description}
	\item  Van Den Kerchove A., Si-Mohammed H., Van Hulle M.M. \& Cabestaing F. (2024) \\
	      \emph{Correcting for ERP latency jitter improves gaze-independent BCI decoding}, \\
	      Journal of Neural Engineering.
	\item Van Den Kerchove A., Libert A., Wittevrongel B. \& Van Hulle M.M.  (2022) \\
	      \emph{Classification of event-related potentials with regularized spatiotemporal LCMV beamforming}, \\
	      Applied Sciences.
	\item Libert A., Van Den Kerchove A., Wittevrongel, B. \& Van Hulle M.M. (2022) \\
	      \emph{Analytic beamformer transformation for transfer learning in motion-onset visual evoked potential decoding}, \\
	      Journal of Neural Engineering.
\end{description}

\subsubsection*{Preprints and journal articles currently under peer review}
\begin{description}
	\item Van Dyck B., Van Den Kerchove, A. \& Van Hulle M.M. (2025)\\
	      \emph{An open-source implementation of a closed-loop electrocorticographic
		      brain-computer interface with Micromed, Fieldtrip, and Psychopy},
	      Biomedical Signal Processing and Control (in review).
	\item Van Den Kerchove A., Meunier J., de Moura, M., Willemssens A., Geurinckx D., Schiettecatte E., Van Damme P., Si-Mohammed H., Cabestaing F., Allart E., \& Van Hulle, M.M. (2025)\\
	      \emph{Visual ERP-based brain-computer interface use with severe physical, speech and eye movement impairments: case studies},
	      Journal of NeuroEngineering and Rehabilitation (in review).
\end{description}

\subsubsection*{Manuscripts in preparation}
\begin{description}
	\item Verovnik B., Hajduk S., Van Den Kerchove A. \& Van Hulle M.M. (2025)\\
	      \emph{Phenotype prediction from resting state fMRI connectivity}.
	\item Van Den Kerchove A., Si-Mohammed H., Cabestaing, F. \& Van Hulle M.M. (2025) \\
	      \emph{Block-Term Tensor Discriminant
		      Analysis for brain-computer interface decoding}.
	\item Van Den Kerchove A., Mlinari\v{c} T., Barinaga Z., Verovnik B., \& Van Hulle, M.M. (2025) \\
	      \emph{EEG-based classification of Alzheimer’s disease and frontotemporal dementia using functional connectivity}.
\end{description}


\subsubsection*{Miscellaneous documents}

\begin{description}
	\item Van Den Kerchove A. (2025) \\
	      \emph{Use cases of tablet-based BCI-AAC in individuals with physical and
		      speech impairment}, experimental protocol.
	\item Van Den Kerchove A., Mirsaeedi M., Van Dyck B. \& Van Hulle M.M. (2025) \\
	      \emph{An event-related potential BCI speller using a wearable, single-channel
		      EEG headset with electrodes on the forehead}, \\
	      Proceedings of the International Work-Conference on Artificial Neural Networks.
	\item Van Den Kerchove A. (2024) \\
	      \emph{A visual brain-computer interface for gaze-free communication},
	      doctoral thesis, KU Leuven \& University of Lille.
	\item Van Den Kerchove A., Libert A., de Borman A., Calvo Merino E. \& Mlin (2022) \\
	      \emph{From problem to Brain-Computer Interface}, course notes, KU Leuven.
	\item Van Den Kerchove A. (2020) \\
	      \emph{Linguistic transcription of EEG responses to sequences of visual stimuli},
	      master's thesis, KU Leuven.

\end{description}


\section*{Awards \& funding}

\cventry{2024}{Clinical trial sponsorship program}{Zeto}{
	Our research project titled ``A gaze-independent Visual Brain-Computer
	Interface for use by patients with limited or no eye control'' won a spot in
	the Zeto Clinical Trial Sponsorship Program, granting us the resources to conduct research
	with the Zeto EEG headset.
}
\cventry{2022}{BCI student club competition}{NeuroTechX}{
	Our NeuroTech-Leuven BrainBrowsR project, a plug-and-play BCI system that lets you control
	social media applications  trough SSVEP-BCI was awarded 90/100 points by
	the jury of the NeuroTechX international student club competition.
}

\cventry{2021 - 2024}{Global Ph.D. Partnership Grant}{KU Leuven and University
	of Lille}{Research funding for a four-year PhD project on EEG-based visual brain-computer interfacing for
	gaze-free communication}


\section*{Conferences and presentations}

\subsubsection*{Presented}

\cventry{2025}{18th International Work-Conference on Artificial Neural Networks}{A Coru\~na}{%
	Poster presentation on single-channel, wearable ERP-BCI speller design.
}
\cventry{2024}{CORTICO Days 2024}{CORTICO, Nancy}{
	Poster presentation on covert visual attention BCIs for patients with
	oculomotor impairment.}
\cventry{2023}{10th International BCI Meeting}{BCI Society, Brussels}{%
	Poster presentation on classifier-based latency estimation for
	gaze-independent BCIs.}
\cventry{2022}{Leuven.AI Scientific Workshop 2022}{Leuven.AI, Leuven}{%
	Poster presentation on (multi-)Kronecker-structured linear discriminant
	analysis for low sample size event-related potential classification.}
\cventry{2022}{CORTICO Days 2022}{CORTICO, Grenoble}{%
	Oral presentation on Kronecker-structured LCMV-beamforming for event-related potential
	classification.}
\cventry{2021}{CORTICO Days 2021}{CORTICO, online}{%
	Oral presentation on a multi-component approach
	to spatiotemporal beamforming decoding of event-related potentials.}

\subsubsection*{Attended}

\cventry{2025}{CORTICO Days 2025}{CORTICO, Lyon}{}
\cventry{2025}{BCI \& Neurotechnology Spring School 2025}{g.tec, online}{}
\cventry{2024}{BCI \& Neurotechnology Spring School 2024}{g.tec, online}{}
\cventry{2023}{Closed Loop Neurotechnologies Autumn School}{NeurotechEU, Lille}{}
\cventry{2023}{CORTICO Days 2023}{CORTICO, Paris}{}
\cventry{2023}{BCI \& Neurotechnology Spring School 2023}{g.tec, online}{}
\cventry{2022}{BCI \& Neurotechnology Masterclass Belgium}{g.tec, online}{}
\cventry{2022}{BCI \& Neurotechnology Spring School 2022}{g.tec, online}{}

\section*{Teaching Experience}

\subsubsection*{Classes}

\cventry{2022 - ongoing}{Teaching assistant \textit{Brain-Computer Interfaces}}{KU
	Leuven}{%
	Teaching exercise sessions in BCI design and signal processing to students in
	the Master of Bioengineering and Advanced Master of Artificial Intelligence.}
\cventry{2022}{Teaching assistant \textit{Fundamentals of Computer Science}}{KU Leuven}{
	Teaching exercise sessions in Python programming and algorithmic reasoning to
	students in the Bachelor of	Engineering Sciences.}

\subsubsection*{Master theses and internships}

\cventry{2024}{Eye-tracker and ERP data fusion for gaze-independent visual
	BCI}{Juliette  Meunier, University of Lille}{}
\cventry{2023 - 2024}{Linear Discriminant Analysis in	the combined
	space-time-frequency domain for BCI decoding}{Lunkyadi Sucipto, KU Leuven}{}
\cventry{2023 - 2024}{Tackling the Midas Touch problem in eye tracking with a
	BCI}{Reniflal Ebenezer Sundaralal, KU Leuven}{}
\cventry{2023 - 2024}{A connectivity-based EEG  analysis of Alzheimer’s Disease
	and Frontotemporal Dementia}{Zoe Barinaga, KU Leuven}{}
\cventry{2023}{Single-trial ERP component latencies as a predictor for the mode
	of visual attention}{Yildiz Dilara Parry, KU Leuven}{}
\cventry{2021 - 2022}{A hybrid P300-gaze BCI alternative for navigating virtual
	spaces}{Gijs Claes, KU Leuven}{}

\subsubsection*{Extracurricular}

\cventry{2022 - 2025}{Project supervisor}{NeuroTech Leuven}{
	Project supervisor and technical advisor for extracurricular
	neurotechnology student projects and a student team competing in the annual NeuroTechX BCI competition and
	other projects related to BCI and neurodegeneration.
}
\cventry{2015 - 2016}{PAL tutor \textit{Principles of Computer Programming}}{KU
	Leuven}{Organizing and teaching peer-assisted learning sessions in Python programming for first year Bachelor of Informatics	students.}

\section*{Professional experience}
\subsubsection*{Professional}

\cventry{2025 - ongoing}{Postdoctoral researcher}{KU Leuven, Laboratory for Neuro- and Psychophysiology}{}
\cventry{2023 - 2024}{Doctoral researcher}{Université de Lille, Research Center for Informatics, Signal Processing and Control Science}{}
\cventry{2021 - 2022}{Doctoral researcher}{KU Leuven, Laboratory for Neuro- and Psychophysiology}{}
\cventry{2019}{Python developer}{Mindspeller}{%
	Application developer in a KU Leuven spin-off that provides neuromarketing services based on
	scientifically validated neuroscience and AI research.}
\cventry{2014 - 2020}{Freelance full-stack web developer}{self-employed}{}

\subsubsection*{Volunteering}
\cventry{2022 - ongoing}{Ambulance EMT}{Belgian Red Cross, FAST	vzw}{}
%2018 - 2019       & \makecell[{{X}}t]{\textbf{Board Secretary}, Scientica Leuven vzw}               \\
%2015 - 2020       & \makecell[{{X}}t]{\textbf{Webmaster}, Wina Leuven vzw and Scientica Leuven vzw} \\
%2013 - ongoing    & \makecell[{{X}}t]{\textbf{Event EMT}, Red Cross}                                \\

\section*{Licenses and certifications}
%\begin{tabularx}{\linewidth}{@{}p{1.2in} X@{}}
%	2022 & \makecell[{{X}}t]{\textbf{Ambulance EMT}, FOD
%	Volksgezondheid}                                                 \\
%	2021 & \makecell[{{X}}t]{\textbf{ICH Good Clinical Practice E6},
%	TransCelerate BioPharma Inc. }                                   \\
%\end{tabularx}

\textbf{ICH Good Clinical Practice}, E6 TransCelerate BioPharma Inc.
\smallskip


\section*{Languages}
\begin{tabularx}{\linewidth}{@{}p{.9in} X@{}}
	Dutch   & Native           \\
	English & Fully proficient \\
	French  & Advanced         \\
	German  & Intermediate     \\
\end{tabularx}




\end{document}
